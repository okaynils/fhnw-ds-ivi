\chapter{Evaluation}

Evaluating the usability and user experience of an interactive visualization is of importance to ensure it meets the needs of its intended audience. This chapter presents an evaluation of the \textit{Birds of Sweden Dashboard} through a qualitative user study involving five participants from a range of backgrounds. The goal is to gain insights into how different users perceive and interact with the dashboard, identifying strengths and areas for improvement.

\section{Methodology}

There are multiple ways of evaluate a piece of software. For this evaluation, I have chosen a qualitative interview approach to gather detailed feedback on the user experience of the dashboard.

\subsection{Selection of Evaluation Method}

A qualitative interview approach was chosen for this evaluation due to its effectiveness in exploring users' thoughts, feelings, and experiences in depth. Unlike quantitative methods, qualitative interviews allow for open-ended responses, providing rich insights into user behavior and preferences. This method is appropriate for measuring user experience in the context of the dashboard, as it enables the identification of usability issues and the collection of detailed feedback that can inform future design enhancements.

\subsection{Participants}

Five users with varying backgrounds, expertise, and interests related to bird observation and data visualization were involved in a qualitative interview:

\begin{enumerate} 
    \item \textbf{Participant 1} -- A 31-year-old bird enthusiast and amateur photographer who enjoys birdwatching during her travels across Sweden.
    \item \textbf{Participant 2} -- A 50-year-old conservationist and hobby ornotihologist.
    \item \textbf{Participant 3} -- A 23-year-old web developer studying Data Science.
    \item \textbf{Participant 4} -- A 24-year-old carpenter, now studying Data Science.
    \item \textbf{Participant 5} -- A 65-year-old retiree who recently started birdwatching as a hobby and is not very tech-savvy.
\end{enumerate}

\section{Interview Questions}

Five open-ended questions were designed to explore key aspects of the user experience. Here I present these questions and their purposes:

\begin{enumerate}
    \item \textbf{How intuitive did you find the navigation and interaction with the dashboard?}

    \textit{Purpose}: To assess the overall usability and whether users can navigate the interface without confusion.

    \item \textbf{What features did you find most useful or engaging, and why?}

    \textit{Purpose}: To identify which elements of the dashboard are most valued by users, informing future enhancements.

    \item \textbf{Were there any difficulties or frustrations you encountered while using the dashboard?}

    \textit{Purpose}: To uncover usability issues or pain points that need to be addressed.

    \item \textbf{How well does the dashboard cater to your specific needs or interests related to bird observation?}

    \textit{Purpose}: To evaluate the dashboard's effectiveness in meeting the diverse needs of different user groups.

    \item \textbf{What improvements or additional features would you suggest for the dashboard?}

    \textit{Purpose}: To gather user-driven ideas for enhancements that could improve satisfaction and usability.
\end{enumerate}

\section{Results}

The interviews were conducted during a one-week period and transcribed for analysis.

The participants' responses are summarized below. Tables are used to present the key points from each interview for clarity.

\begin{table}[H]
    \centering
    \begin{tabular}{p{3cm} | p{12cm}}
        \hline
        \textbf{Participant} & \textbf{Responses} \\
        \hline
        \multicolumn{2}{l}{\textbf{Question 1: Navigation and Interaction Intuitiveness}} \\
        \hline
        Participant 1 & Found the dashboard mostly intuitive. Initially unsure about how to interact with map points but appreciated the straightforward species selection once she became accustomed. \\
        \hline
        Participant 2 & Navigated easily due to familiarity with similar tools; praised the clean, professional interface but noted the potential for advanced navigational aids for detailed searches. \\
        \hline
        Participant 3 & Found the interface user-friendly and intuitive, commended the responsive design, though noted minor inconsistencies in menu transitions. \\
        \hline
        Participant 4 & Experienced slight confusion with sidebar updates not being immediately noticeable; suggested clearer visual cues for when data changes. \\
        \hline
        Participant 5 & Initially overwhelmed but adapted quickly after a brief exploration. Appreciated the clear dropdown menu once located but felt it could be made more visible. \\
        \hline
        \multicolumn{2}{l}{\textbf{Question 2: Most Useful or Engaging Features}} \\
        \hline
        Participant 1 & Enjoyed the species images and detailed descriptions, particularly appreciating the ability to locate her favorite birds' habitats. \\
        \hline
        Participant 2 & Valued the observation distribution visualizations, which were highly useful for understanding migration patterns. \\
        \hline
        Participant 3 & Was particularly impressed by the interactive maps and their hover effects, which dynamically updated related data. \\
        \hline
        Participant 4 & Found the state-level observation map valuable for identifying regions that might require conservation efforts. \\
        \hline
        Participant 5 & Loved the ability to explore species and their habitats through the interactive maps, finding the feature highly educational and engaging. \\
        \hline
    \end{tabular}
    \caption{Participant Responses to Questions 1 and 2}
    \label{tab:responses1}
\end{table}


\begin{table}[H]
    \centering
    \begin{tabular}{p{3cm} | p{12cm}}
        \hline
        \textbf{Participant} & \textbf{Responses} \\
        \hline
        \multicolumn{2}{l}{\textbf{Question 3: Difficulties or Frustrations}} \\
        \hline
        Participant 1 & Struggled to hover over small map points; suggested larger clickable areas or a zoom-in feature to make interactions easier. \\
        \hline
        Participant 2 & Felt the lack of advanced filtering options (e.g., by date range or observation count) limited deeper analyses. \\
        \hline
        Participant 3 & Noted issues with mobile responsiveness, particularly on tablets where some menu elements were misaligned. \\
        \hline
        Participant 4 & Encountered challenges when using assistive technology; highlighted the importance of improving accessibility features. \\
        \hline
        Participant 5 & Was initially confused about the species selection due to the dropdown’s placement and size but adapted after finding it. \\
        \hline
        \multicolumn{2}{l}{\textbf{Question 4: Catering to Specific Needs or Interests}} \\
        \hline
        Participant 1 & Found the dashboard catered well to her interest in discovering new birdwatching spots and learning about bird habitats. \\
        \hline
        Participant 2 & Appreciated the geographical data but desired more detailed analytics and tools to support academic research needs. \\
        \hline
        Participant 3 & Found the dashboard inspiring for his studies in design and praised its application of interaction design principles. \\
        \hline
        Participant 4 & Saw potential for using the dashboard in conservation planning but wanted more data layers, such as vegetation and climate overlays. \\
        \hline
        Participant 5 & Enjoyed learning about bird species and habitats, which added excitement to her beginner birdwatching efforts. \\
        \hline
        \multicolumn{2}{l}{\textbf{Question 5: Suggestions for Improvement}} \\
        \hline
        Participant 1 & Suggested adding bird call sounds for a richer experience and a tutorial to guide new users through the dashboard. \\
        \hline
        Participant 2 & Recommended incorporating time-series data to observe seasonal trends and changes in bird populations over the years. \\
        \hline
        Participant 3 & Proposed improving mobile responsiveness and adding gesture controls to enhance usability on touchscreens. \\
        \hline
        Participant 4 & Highlighted the need for accessibility features like screen reader support, high-contrast modes, and keyboard navigation. \\
        \hline
        Participant 5 & Suggested simplifying the interface with larger text and icons to make it more beginner-friendly and accessible. \\
        \hline
    \end{tabular}
    \caption{Participant Responses to Questions 3, 4, and 5}
    \label{tab:responses2}
\end{table}


\section{Analysis}

The qualitative interviews revealed some interesting key themes and insights. Here, I summarized them into the following five categories:

\subsection{Usability and Navigation}

Most participants found the dashboard intuitive, particularly those with technical backgrounds. However, less tech-savvy users like Sara experienced initial confusion, indicating a need for improved onboarding or tutorials. The visibility of interactive elements like the species dropdown could be enhanced to aid navigation.

\subsection{Engaging Features}

Interactive maps and species information were consistently highlighted as valuable features. Visual representations and the inclusion of images contributed positively to user engagement, aligning with the importance of visual elements in interaction design.

\subsection{Usability Issues and Frustrations}

Common difficulties included:

\begin{itemize}
    \item \textbf{Hovering over Data Points}: Small interactive areas made it challenging to access tooltips, supporting the need to apply Fitts's Law considerations.
    \item \textbf{Accessibility}: One user specifically emphasized the lack of support for assistive technologies and named the findings from the literature on accessible visualization.
    \item \textbf{Mobile Optimization}: Issues on tablets and mobile devices suggest the dashboard is not fully responsive, limiting usability across devices.
\end{itemize}

\subsection{Meeting User Needs}

Participants with specific interests, such as conservation and research, found the dashboard partially met their needs but desired more advanced features like filtering options and additional data layers. This indicates opportunities to expand functionality to cater to specialized user groups.

\subsection{Suggestions for Improvement}

Key recommendations included:

\begin{itemize}
    \item \textbf{Enhancing Accessibility}: Implementing features like screen reader support and high-contrast modes.
    \item \textbf{Mobile Responsiveness}: Optimizing the dashboard for use on tablets and smartphones.
    \item \textbf{Interactive Enhancements}: Adding bird sounds, gesture controls, and tutorials to improve engagement and usability.
    \item \textbf{Advanced Features}: Incorporating time-series data and advanced filtering to support research and conservation efforts.
\end{itemize}