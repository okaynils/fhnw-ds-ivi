\chapter{Evaluation}

\section{Introduction}

Evaluating the usability and user experience of an interactive visualization is crucial to ensure it meets the needs of its intended audience. This chapter presents an evaluation of the \textit{Birds of Sweden Dashboard} through a qualitative user study involving five participants from diverse backgrounds. The goal is to gain insights into how different users perceive and interact with the dashboard, identifying strengths and areas for improvement.

\section{Methodology}

\subsection{Selection of Evaluation Method}

A qualitative interview approach was chosen for this evaluation due to its effectiveness in exploring users' thoughts, feelings, and experiences in depth. Unlike quantitative methods, qualitative interviews allow for open-ended responses, providing rich insights into user behavior and preferences. This method is appropriate for measuring user experience in the context of the dashboard, as it enables the identification of usability issues and the collection of detailed feedback that can inform future design enhancements.

\subsection{Participants}

Five users with varying backgrounds, expertise, and interests related to bird observation and data visualization were involved in a qualitative interview:

\begin{enumerate} 
    \item \textbf{Anna} -- A 31-year-old bird enthusiast and amateur photographer who enjoys birdwatching during her travels across Sweden.
    \item \textbf{Lasse} -- A 50-year-old conservationist and hobby ornotihologist.
    \item \textbf{Noah} -- A 23-year-old web developer studying Data Science.
    \item \textbf{Boran} -- A 24-year-old carpenter, now studying Data Science.
    \item \textbf{Agnetha} -- A 65-year-old retiree who recently started birdwatching as a hobby and is not very tech-savvy.
\end{enumerate}

\section{Interview Questions}

Five open-ended questions were designed to explore key aspects of the user experience:

\begin{enumerate}
    \item \textbf{How intuitive did you find the navigation and interaction with the dashboard?}

    \textit{Purpose}: To assess the overall usability and whether users can navigate the interface without confusion.

    \item \textbf{What features did you find most useful or engaging, and why?}

    \textit{Purpose}: To identify which elements of the dashboard are most valued by users, informing future enhancements.

    \item \textbf{Were there any difficulties or frustrations you encountered while using the dashboard?}

    \textit{Purpose}: To uncover usability issues or pain points that need to be addressed.

    \item \textbf{How well does the dashboard cater to your specific needs or interests related to bird observation?}

    \textit{Purpose}: To evaluate the dashboard's effectiveness in meeting the diverse needs of different user groups.

    \item \textbf{What improvements or additional features would you suggest for the dashboard?}

    \textit{Purpose}: To gather user-driven ideas for enhancements that could improve satisfaction and usability.
\end{enumerate}

\section{Results}

The participants' responses are summarized below. Tables are used to present the key points from each interview for clarity.

\begin{table}[H]
    \centering
    \begin{tabular}{p{3cm} | p{12cm}}
        \hline
        \textbf{Participant} & \textbf{Responses} \\
        \hline
        \multicolumn{2}{l}{\textbf{Question 1: Navigation and Interaction Intuitiveness}} \\
        \hline
        Anna & Found the dashboard mostly intuitive, appreciated the straightforward species selection, but initially unsure about clicking on map points for details. \\
        \hline
        Lasse & Navigated easily due to familiarity with similar tools; found the interface clean and professional. \\
        \hline
        Noah & As a developer, found the interface user-friendly and appreciated the responsive design. \\
        \hline
        Boran & Experienced slight confusion with the sidebar updates; suggested clearer visual cues when data changes. \\
        \hline
        Agnetha & Felt overwhelmed at first but became comfortable after a brief exploration; found the dropdown helpful once located. \\
        \hline
        \multicolumn{2}{l}{\textbf{Question 2: Most Useful or Engaging Features}} \\
        \hline
        Anna & Loved the species images and details; enjoyed seeing where her favorite birds are commonly found. \\
        \hline
        Lasse & Valued the data visualization of observation distributions; useful for his research on migration patterns. \\
        \hline
        Noah & Impressed by the interactive maps; appreciated the hover effects and dynamic updates. \\
        \hline
        Boran & Found the state observations map valuable for identifying regions needing conservation efforts. \\
        \hline
        Agnetha & Enjoyed the ability to select different species and see where they are located; found it engaging and educational. \\
        \hline
    \end{tabular}
    \caption{Participant Responses to Questions 1 and 2}
    \label{tab:responses1}
\end{table}

\begin{table}[H]
    \centering
    \begin{tabular}{p{3cm} | p{12cm}}
        \hline
        \textbf{Participant} & \textbf{Responses} \\
        \hline
        \multicolumn{2}{l}{\textbf{Question 3: Difficulties or Frustrations}} \\
        \hline
        Anna & Had trouble hovering over small data points; suggested larger clickable areas or zoom functionality. \\
        \hline
        Lasse & Noted a lack of advanced filtering options, such as filtering by date or observation count. \\
        \hline
        Noah & Pointed out that the dashboard is not fully optimized for mobile devices; experienced issues on her tablet. \\
        \hline
        Boran & Found it challenging to use the dashboard with assistive technology; suggested improvements for accessibility. \\
        \hline
        Agnetha & Initially confused about how to select a species; the dropdown was not immediately apparent to her. \\
        \hline
        \multicolumn{2}{l}{\textbf{Question 4: Catering to Specific Needs or Interests}} \\
        \hline
        Anna & Felt the dashboard catered well to her interest in discovering new birdwatching spots. \\
        \hline
        Lasse & Appreciated the geographical data but desired more detailed analytics for academic purposes. \\
        \hline
        Noah & Found it inspiring for design ideas; appreciated the integration of interaction design principles. \\
        \hline
        Boran & Saw potential for using the dashboard in conservation planning but needed more data layers. \\
        \hline
        Agnetha & Enjoyed learning about different species; helped her as a beginner in birdwatching. \\
        \hline
        \multicolumn{2}{l}{\textbf{Question 5: Suggestions for Improvements}} \\
        \hline
        Anna & Recommended adding bird call sounds to enhance the experience; suggested a tutorial for new users. \\
        \hline
        Lasse & Suggested incorporating time-series data to observe changes over seasons or years. \\
        \hline
        Noah & Proposed making the dashboard more responsive for mobile use; adding gesture controls. \\
        \hline
        Boran & Emphasized the need for accessibility features like screen reader support and high-contrast modes. \\
        \hline
        Agnetha & Wanted larger text and icons; suggested simplifying the interface for ease of use. \\
        \hline
    \end{tabular}
    \caption{Participant Responses to Questions 3, 4, and 5}
    \label{tab:responses2}
\end{table}

\section{Analysis}

The qualitative interviews revealed several key themes and insights:

\subsection{Usability and Navigation}

Most participants found the dashboard intuitive, particularly those with technical backgrounds. However, less tech-savvy users like Sara experienced initial confusion, indicating a need for improved onboarding or tutorials. The visibility of interactive elements like the species dropdown could be enhanced to aid navigation.

\subsection{Engaging Features}

Interactive maps and species information were consistently highlighted as valuable features. Visual representations and the inclusion of images contributed positively to user engagement, aligning with the importance of visual elements in interaction design.

\subsection{Usability Issues and Frustrations}

Common difficulties included:

\begin{itemize}
    \item \textbf{Hovering over Data Points}: Small interactive areas made it challenging to access tooltips, supporting the need to apply Fitts's Law considerations.
    \item \textbf{Accessibility}: Users like Olle emphasized the lack of support for assistive technologies, echoing the findings from the literature on accessible visualization.
    \item \textbf{Mobile Optimization}: Issues on tablets and mobile devices suggest the dashboard is not fully responsive, limiting usability across devices.
\end{itemize}

\subsection{Meeting User Needs}

Participants with specific interests, such as conservation and research, found the dashboard partially met their needs but desired more advanced features like filtering options and additional data layers. This indicates opportunities to expand functionality to cater to specialized user groups.

\subsection{Suggestions for Improvement}

Key recommendations included:

\begin{itemize}
    \item \textbf{Enhancing Accessibility}: Implementing features like screen reader support and high-contrast modes.
    \item \textbf{Mobile Responsiveness}: Optimizing the dashboard for use on tablets and smartphones.
    \item \textbf{Interactive Enhancements}: Adding bird sounds, gesture controls, and tutorials to improve engagement and usability.
    \item \textbf{Advanced Features}: Incorporating time-series data and advanced filtering to support research and conservation efforts.
\end{itemize}

\section{Conclusion}

The evaluation provided valuable insights into the user experience of the \textit{Birds of Sweden Dashboard}. While the dashboard effectively engages users and offers useful features, there are areas for improvement, particularly in accessibility, mobile optimization, and user guidance. Addressing these issues will enhance usability for a broader audience, ensuring the dashboard meets the needs of diverse users ranging from casual bird enthusiasts to professional researchers.

Future development should focus on:

\begin{itemize}
    \item \textbf{Improving Accessibility}: Integrate support for assistive technologies and design for inclusivity.
    \item \textbf{Optimizing for Mobile Devices}: Ensure responsive design principles are applied for seamless use across all devices.
    \item \textbf{Enhancing Interactivity}: Implement user suggestions like auditory feedback and gesture controls to enrich the interactive experience.
    \item \textbf{Expanding Functionality}: Add advanced features like time-series data and detailed filtering to cater to specialized user needs.
\end{itemize}
